\documentclass{scrartcl}

\usepackage[hidelinks]{hyperref}
\usepackage[none]{hyphenat}
\usepackage{setspace}
\doublespace

\title{How does introversion influence the ability of game developers to engage effectively with scrum?}

\subtitle{COMP150 - Agile Development Practice}

\author{1803778}
\date{October 2018}

\begin{document}
    \pagenumbering{gobble}
    \maketitle
    
    Beliefs, Practices, and Personalities of Software Engineers \cite{7809477}.
    Supporting agile team composition \cite{5071413}.
    Maturity level in Software Development \cite{7965314}.
    The Affect of Software Developers \cite{7166106}.
    The Basic Research of Human Factor Analysis \cite{4723148}
    The Social Nature of Agile Teams \cite{4599518}
    Belbin Team Roles \cite{Belbin01}
    
    \newpage
    \subsection{Introduction}
    The Waterfall Development Cycle approach is often favoured among introverts, as it allows for them to not have the necessity of communications with the team. This is due to the fact that the waterfall model is more task based. Meaning that members within the team can easily work on their own if they so wish. The Agile Development Cycle on the other hand is more directed towards a team based environment. So communication is key in order to reach the end goal.
    
    \subsection{Understanding Personality Types}
    In game development, the key to getting a fully functional team working, is to understand who the team are as individuals and how they act when in a working environment. So, if a team has a nice balance of different personality types who can easily support each other, they tend to perform much better than a team that is more homogeneous\cite{5071413}. Results from the paper suggest a study was conducted on 69 software development teams, which proved that managing personality traits has a significant positive impact on the overall performance. Also, if an individual has a been given a role that matches their natural preference.Then it will create a better work ethic for the whole team. So in relation to introversion, we can infer that an introverted person would be more happy with a role like the 'Plant'\cite{Belbin01}. This is due to the similarities between the two. As introverts like to do their own things and have that feeling of creative freedom when doing them. This is why introverts make the best Plants and always work well with this assigned role.
    
    \subsection{The Risks Faced With Introversion}
    In some cases, poorly made role assignments can be made. This type of ill-management can lead to a set of personality clashes within the team. Which will then lead to the threatening of the whole project in the long run. For example, a scenario where this kind of behaviour might occur is in a larger scale project where documentation becomes a near necessity and could become affected if the team experience personality clashes. This is because they'd lose that sense of human interaction. Meaning that they'd lose vital tacit knowledge\cite{5071413}. But if problems within the team persist, then actions needs to be taken. The right course of action would be ejecting the team member. As long as the reason is justified. For instance, they might not have engaged with any of the team members or leadership to get feedback for what to do/improve upon.
    
    \subsection{Agile In Game Development}
    \pagenumbering {Roman}
    
    \newpage
    \newpage
    
\bibliography{References}
\bibliographystyle{acm}

\end{document}
